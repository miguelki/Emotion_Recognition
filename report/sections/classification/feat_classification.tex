\phantomsection
\chapter{Feature classification}
\label{chap:classification}

\noindent Classification is done through machine learning algorithms. Machine learning, being a branch of Artificial Intelligence, aims to helps AI systems improve their performances by learning from their environment. Indeed, the knowledge necessary to build a robust and intelligent system can not always be built-in or explained to it. The solution to this problem is to make the system learn this knowledge through examples, and apply it to similar situations so it can perform relevant actions without the need of human intervention. 
\newline

\noindent More specifically, machine learning can be described as a way to develop and implement algorithms taking empirical data as input, and processing these values in order to find links between them. The output will then be used by the system to compute the appropriate action or behaviour. In order to achieve that, the system needs to learn key characteristics from a training dataset given as example or obtained through past experience. It will then study this observable data, and build a model based on it. The system will then use this model to infer actions depending on new data it will get as input.
\newline

\noindent However, machine learning is not only about computing a database and relying on it for every possible situation. In a changing environment, the system needs to know how to learn from these changes and adapt itself.
\newline

\noindent There are many kinds of machine learning algorithms, with their own specificities and level of abstraction.
\newline

\subsection{Supervised learning}

\vspace{\baselineskip}
\noindent bla
\newline

\subsection{Unsupervised learning}

\vspace{\baselineskip}
\noindent bla
\newline

\noindent concept of clustering
\newline


