\phantomsection
\chapter{Viola-Jones}

\noindent Viola-Jones algorithm is a visual object detection framework that is capable of processing images extremely rapidly while achieving high detection rates. It is based on a new image representation called "Integral Image" which allows the features used to be computed very quickly. It is also based on a learning algorithm, based on AdaBoost that gives extremely efficient classifiers. And it is also based on a method for combining classifiers in "cascade"; it allows to discard quickly the background of the image and to focus on the promising object-like regions \cite{VIO01}.
\newline

\phantomsection
\section{Haar features}

\noindent The features used by Viola and Jones are called Haar features and are based on Haar wavelets. Haar wavelets are single wavelength square waves (one high interval and one low interval). In two dimensions, a square wave is a pair of adjacent rectangles: one light and one dark. The actual rectangle combinations used for visual object detection are not true Haar wavelets. Instead, they contain rectangle combinations better suited to visual recognition tasks. That is because of this difference that these features are called Haar features, or Haarlike features, rather than Haar wavelets \cite{HEW07}.
\newline

\noindent For example, the figure~\ref{haar_features_first_2_stage} shows the first two Haar features in the original Viola-Jones cascade \cite{HEW07}. In the figure~\ref{haar_features_early_stage} , it is an example of a early stage in the Haar cascade. Each black and white patch represents a feature that the algorithm hunts for in the image \cite{HAR12}.
\newline

\begin{figure}[!h]
\begin{center}
\noindent \includegraphics[scale=0.9]{figures/haar_features_first_2_stage} 
\newline
\caption{Example of the first two Haar features}
\label{haar_features_first_2_stage}
\end{center} 
\end{figure}

\begin{figure}[!h]
\begin{center}
\noindent \includegraphics[scale=0.5]{figures/haar_features_early_stage} 
\newline
\caption{Example of an early stage in the Haar cascade}
\label{haar_features_early_stage}
\end{center} 
\end{figure}

\noindent The presence of a Haar feature is determined by subtracting the average dark-region pixel value from the average light-region pixel value. if the difference is above a threshold, that feature is said to be present and then it can go on to the next stage \cite{HEW07}. There is about 20-30 different stages. The first stage is a very coarse scan of the image. Stage 2 gets a little more detailed, stage 3 is a harder test to pass, stage 4 is even harder and it goes on and on. More it goes further into the cascade, more the features get increasingly complex and larger. It also takes more time to compute \cite{HAR12}.
\newline

\noindent For example, the figure~\ref{haar_feature_later_stage} shows the later stage in the Haar cascade where many more patterns of black and white rectangles need to match the candidate image \cite{HAR12}.
\newline

\begin{figure}[!h]
\begin{center}
\noindent \includegraphics[scale=0.8]{figures/haar_feature_later_stage} 
\newline
\caption{Example of the later stage in the Haar cascade}
\label{haar_feature_later_stage}
\end{center} 
\end{figure}

\noindent Three kinds of feature are used by Viola and Jones. The value of a two-rectangle feature is the difference between the sum of the pixels within two rectangular regions. The regions have the same size and shape and are horizontally or vertically adjacent. A three-rectangle feature computes the sum within two outside rectangles subtracted from the sum in a center rectangle. Finally a four-rectangle feature computes the difference between diagonal pairs of rectangles \cite{VIO01}.
\newline

\noindent For example, the figure~\ref{haar_feature_description} shows the different kinds of rectangle features used by the Viola-Jones algorithm. Two-rectangle features are shown in (A) and (B). Figure (C) shows a three-rectangle feature, and (D) a four-rectangle feature \cite{VIO01}.
\newline

\begin{figure}[!h]
\begin{center}
\noindent \includegraphics[scale=0.6]{figures/haar_feature_description} 
\newline
\caption{Example of the different kinds of rectangle features}
\label{haar_feature_description}
\end{center} 
\end{figure}

\phantomsection
\section{Integral image}

\noindent bla bla bla
\newline

\phantomsection
\section{Weak classifiers}

\noindent bla bla bla
\newline

\phantomsection
\section{AdaBoost}

\noindent bla bla bla
\newline

\phantomsection
\section{Test set and training}

\noindent bla bla bla
\newline