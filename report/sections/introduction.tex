%\newpage
  \begin{titlepage}
    \vspace*{\fill}
      \part{Introduction}
    \vspace*{\fill}
  \end{titlepage}

%\chapter{Introduction}\label{ch:introduction}
\phantomsection
\chapter*{Contents}
<<<<<<< HEAD
First, this project is motivated by analyzing the need of robust facial expression recognition systems for various applications. Then already existing algorithms will be studied to choose one that is basic but effective in order to improve it. In our last part, we will formulate the problem.
\newpage
=======
First, this project is motivated by analyzing the need of robust facial expression recognition systems for various applications. Then already existing algorithms will be studied to choose one that is basic but effective in order to improve it. In the last part, the problem will be formulated.
>>>>>>> Avancement intro rapport

\phantomsection
\chapter{Motivation}

A facial expression is a visible manifestation of the effective state, cognitive activity, intent, personality, and psychopathology of a person \cite{DON99}; facial expressions play a significant role in human dialogue and in human interaction. Indeed, facial expressions carry other information than speech and humans relay on that for their interaction. Facial expressions have a considerable effect on a listening interlocutor; the facial expression of a speaker accounts for about 55 percent of the effect, 38 percent of the latter is conveyed by voice intonation and 7 percent by the spoken words \cite{PAN00}.
\newline

\noindent Since antiquity, searchers have been interested in emotion and more particularly in emotion recognition. But one of the important works on facial expression analysis that has a direct relationship to the modern day science of automatic facial expression recognition was the work done by Charles Darwin \cite{BET12}. In 1872, Darwin wrote a treatise that established the general principles of expression and the means of expressions in both humans and animals \cite{DAR04}. He also grouped various kinds of expressions into similar categories. This was the beginnings of facial expression recognition.
\newline

\noindent Now, with the emergence of new technologies and computers, researchers have put their interests on automatic facial expression recognition by computers. Because facial expressions are important in human interaction, this will add many possibilities in the domain of Human-Machine Interaction. Indeed with emotion recognition, the computers can be more responsive to the users' emotions and this way, interaction will not be as cold as the one we know. 
\newline

\noindent Another domain that is really interested in facial expression recognition is robotics. With the advances in robotics, now robots tend to mimic human emotion and to react as closely as humans as possible, especially for the humanoid robots. But because robots become a more and more important part in our lives, they need to understand and recognize human emotions.
\newline

\noindent But there are various other domains where emotion recognition can be used: Telecommunications, Behavioral Science, Video Games, Animations, Psychiatry, Automobile Safety, Affect sensitive music juke boxes and televisions, Educational Software, etc \cite{BET12}.
\newline

\noindent A lot of real time applications have already been created. For example, Bartlett et al. have successfully used their face expression recognition system to develop an animated character that mirrors the expressions of the user (called the CU Animate) \cite{BAR03}. They have also been successful in deployed the recognition system on Sony's Aibo Robot and ATR's RoboVie \cite{BAR03}. Another interesting application has been demonstrated by Anderson and McOwen, called the "EmotiChat" \cite{AND06}. It is a chatroom where users can log in and start chatting. Their facial expression recognition system is connected to the chat and convert into emoticones the facial expression of the users. Because facial expression recognition system becomes more and more robust and more and more reliable, lot of innovative applications will turn out.
\newpage

\phantomsection
\chapter{Existing systems}

\phantomsection
\section{Facial Expression Recognition system}

\noindent Facial Expression Recognition is a system that allows to recognize emotion on a human face in an automatic way. Facial Expression Recognition can be based on images or on videos; it can be real time or not. Most of the time, searchers use images of human faces and try to recognize the emotion. But this can also be done in real time by using video. While the person is having and expressing his emotions, the Facial Expression Recognition system is analyzing the video and detect in real time the emotion of the person.
\newline

\noindent In both cases, Facial Expression Recognition process is composed as follows:


\noindent \includegraphics[scale=0.6]{figures/facial_expression_recognition_process}

<<<<<<< HEAD
\noindent Before developing our facial expression recognition project, it is important to know what already exist; the state of the art of facial expression recognition system. In this chapter, we will give an overview of the existing systems before we decide on a system for our project.


=======
The first step of the process is "Image Acquisition". The images used for Facial Expression Recognition can be static images or a image sequences. Image sequences gives more information about the facial expression, as the steps in the movement of the muscles of the face to get to the facial expression. Concerning the static images, most of the time, 2D grey-scale images are used for automatic Facial Expression Recognition system. But in the future, more and more systems will potentially use more and more color images. First because the technologies capable of capturing images or image sequences become more affordable. And then, because colors can give more information on emotion as blushing \cite{CHI03}.

The second step is 'Face Detection'. 'Face Detection' is a necessary step before the 'Pre-Processing' one. Indeed, in an static image and even more in an image sequence, the need of detecting the face is obvious. Once the face has been detecting, all the other information in the image can be deleted. Only the face is needed. In a Facial Expression Recognition system working in real-time with image sequences, the face needs to be detected but also to be tracked. The most used and famous algorithm capable of doing that is the Viola-Jones Algorithm.


Before developing a facial expression recognition project, it is important to know what already exist; the state of the art of facial expression recognition system. In this chapter, an overview will be given of the existing systems before to decide on a system for the project.
>>>>>>> Avancement intro rapport
