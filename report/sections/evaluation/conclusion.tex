\phantomsection
\chapter{Conclusion}
\label{chap:ccl}
  
\phantomsection
\section{Theoretical framework}

\vspace{\baselineskip}
\noindent viola jones
\newline

\noindent lbp
\newline

\noindent svm
\newline

\phantomsection
\section{Results}

\vspace{\baselineskip}
\noindent bla bla bla
\newline

\phantomsection
\section{Improvements}

\subsection{Local Binary Patterns}

\vspace{\baselineskip}
\noindent One of the way to improve the LBP operator of this system is to add weights for each of the 42 regions of the face, as seen in Chapter~\ref{chap:lbp}. The LBP operator used in this system is already far from the basic LBP operator; it is a uniform circular LBP operator. Even though the results obtained with this operator are quite good with an accuracy of $ 66.67\% $, there is still room for improvement. Weighting the regions of the face will have an impact on the computed histogram, hence on the resulting feature vector ready for classification. \newline

\noindent The face image is divided in 42 regions ($ 7 $ rows $\times$ $ 6 $ columns), as seen in Chapter~\ref{chap:lbp}, and mentioned in \ref{GAN08}. The weights are however not applied in the same way as in \ref{GAN08}. Figure~\ref{lbp_region_weight} given in the chapter~\ref{chap:lbp} shows a face image cropped in a specific way. Face images extracted from the KDEF database with Viola-Jones algorithm, on which this system is based, are not the same. Indeed, they still have some background around them, and the algorithm extracts faces with surrounding hair. Thus, border regions are less important than those containing the ROI of the face (eyes, nose, mouth). Figure~\ref{implementation_weight_example} shows an example of the division into regions of face images from the KDEF database used by this system.
\newline

\begin{figure}[!h]
\begin{center}
\noindent \includegraphics[scale=0.3]{figures/implementation_weight_example} 
\newline
\caption{Example of division into regions of face images from the KDEF database}
\label{implementation_weight_example}
\end{center} 
\end{figure}

\noindent Assigned weights are thus adapted to extracted faces, so they do not match those in Figure~\ref{lbp_region_weight} anymore. Updated weights are assigned as in Figure~\ref{implementation_lbp_weight}. As it is shown in Figure~\ref{implementation_lbp_weight}, borderline weights all have a value of 1. It is because these regions are considered as background and are not important when performing facial expression recognition. All weights of regions characterizing the face are at least equal to 2, while weights for regions containing ROI are equal to 4. The regions that have highest weights hence are the region of the eyes and of the mouth.
\newline

\begin{figure}[!h]
\begin{center}
\noindent \includegraphics[scale=0.5]{figures/implementation_lbp_weight} 
\newline
\caption{Weights assignment used in this system}
\label{implementation_lbp_weight}
\end{center} 
\end{figure}

\noindent GIVE RESULTS WITH WEIGHTS APPLIED
\newline

\noindent Another way to improve the LBP operator would be to use one with larger scale hence modify the radius of the circular operator; for example, $ LBP_{12,2.5}^{u^2} $ or $ LBP_{16,4.0}^{u^2} $ (with $ P = 12 $ and $ R = 2.5 $ or with $ P = 16 $ and $ R = 4.0 $). This implies to use the bilinear interpolation because sampling points do not fall exactly on pixels, as seen in Chapter~\ref{chap:lbp}. However, using bilinear interpolation is more computationally expensive. A good compromise has to be found between computation time and accuracy rate.
\newline

\subsection{Combination of feature extraction methods}

\vspace{\baselineskip}
\noindent To improve the accuracy of LBP feature extraction method, another feature extraction method can be used and combined with it. 
\newline

\noindent  For example, a method has been proposed by Liao et al. \cite{LIA09}, where they combine LBP and Gabor filter. It is called Dominant Local Binary Patterns (DLBP), and is robust against change of lighting, image rotation and image noise.  It works by using the most recurrent patterns of the LBP method to obtain more information on the texture. It also uses the Gabor method to add global texture information to one already obtained by LBP. It works based on the circularly symmetric Gabor filter responses \cite{LIA09}. Figure~\ref{combination_lbp_gabor} contains two face images of the YaleB face database , and shows the robustness of the combination against change in lighting. Image (a) is the original image with different lighting conditions, and image (b) is the preprocessed image with the Gabor wavelets. Image (c) is the same image, mapped with the LBP operator. Finally, image (d) is the preprocessed image with the combination of the two methods \cite{GOH11}.
\newline

\begin{figure}[!h]
\begin{center}
\noindent \includegraphics[scale=1]{figures/combination_lbp_gabor} 
\newline
\caption{2 face images of the YaleB face database (a) when processed with Gabor wavelets (b), LBP (c), and Gabor wavelets+LBP (d)}
\label{combination_lbp_gabor}
\end{center} 
\end{figure}

\noindent  LBP has also been combined with the Scale Invariant Feature Transform descriptor (SIFT), which is a ROI descriptor. This descriptor is robust against image rotation, image translations, scaling and lighting variations. Heikkila et al. introduced a combination of the SIFT descriptor with the LBP operator \cite{HEI09}.
\newline
